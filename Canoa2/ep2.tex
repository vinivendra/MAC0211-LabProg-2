\documentclass[a4paper,11pt]{article}
\usepackage[T1]{fontenc}
\usepackage[utf8]{inputenc}
\usepackage{lmodern}
\usepackage[brazil]{babel}
\usepackage{listings}
\lstset{language=C}
\usepackage{tikz}
\usepackage{graphicx}
\usepackage{amsmath}
\usepackage{amsthm}
\usepackage{amsfonts}
\PassOptionsToPackage{usenames,dvipsnames,svgnames}{xcolor}  
\usetikzlibrary{arrows,positioning,automata}
\begin{document}

\begin{center}{\Large \bf Documenta\c{c}ão EP2-A Canoagem \\ }\end{center}
\begin{center}{\Large \bf Manual do Desenvolvedor\\ }\end{center}

\begin{center}
{
Fellipe Souto Sampaio\footnote{Número USP: 7990422 e-mail: fellipe.sampaio@usp.com}
Gervásio Protásio dos Santos Neto \footnote{Número USP: 7990996 e-mail: gervasio.neto@usp.br}
Vinícius Jorge Vendramini \footnote{Número USP: 7991103 e-mail: vinicius.vendramini@usp.br}
}

\end{center}

\begin{center}
MAC 0211 Laboratório de Programa\c{c}ão I \\
Prof. Kelly Rosa Braghetto \\
             
\end{center}

\begin{center}
Instituto de Matemática e Estatística - IME USP \\
 Rua do Matão 1010 \\
 05311-970\, Cidade Universitária, São Paulo - SP \\
\end{center}

\newpage

\section{Introdu\c{c}ão}
Esta documenta\c{c}ão apresenta uma completa descri\c{c}ão do funcionamento do exercício programa 2-A e seus algoritmos, divido através de sua implementa\c{c}ão nos .c e .h .

\section{Pixel}
Para modelar o funcionamento da grade decidiu-se implementar um matriz de tipo pixel. Nesta estrura existe dois campos, \textit{velocidade} e \textit{tipo} que guardam a velocidade e o tipo do ponto respectivamente. Para lidar com estar estrutura foi implemetado as seguinte fun\c{c}ões:
\begin{enumerate}
\item[•]{velocidade - Retorna a velocidade de um ponto.}
\item[•]{tipo - Retorna o tipo de um ponto.} 
\item[•]{setaVelocidade - Muda a velocidade do ponto para o valor fornecido.} 
\item[•]{setaTipo - Muda a natureza do ponto para o tipo fornecido.} 

\end{enumerate}


\section{Rio}
Esta interface lida com toda opera\c{c}ão dos pontos, desde a gera\c{c}ão das margens, ilhas, velocidade da água entre outras. Temos duas fun\c{c} principais, primeiraLinha e proximaLinha, as quais estão disponíveis para outras interfaces. Seus métodos internos são chamados para constru\c{c}ão das linhas, cada qual exercendo uma funcionalidade própria. O fluxo de funcionamento das duas fun\c{c} está descrito nos anexos 1 e 2 respectivamente, onde cada vértice representa um método aplicado, o valor das aretas é a ordem em que são chamadas e por fim a orienta\c{c}ão do grafo informa o fluxo de informa\c{c}ões trocadas entre a fun\c{c} chamadora e a chamada.\\

\subsection{primeiraLinha}
Esta fun\c{c}ão tem a funcionalidade de criar a primeira linha da grade, que servirá como linha base para linhas subsequentes. Chama-se o método \textit{aleatorizaPrimeiraMargem} para definir tamanhos aleatórios para as margens esquerda e direita, dentro do limite estipulado pela variável limiteMargens. Após isso chama-se o método \textit{insereIlha} que tem o trabalho de inserir um obstáculo na por\c{c}ão central do rio de acordo com a probabilidade inserida pelo usuário. Um valor default definido em main.c é atribuido a probabilidade e a distância mínima entre ilhas, caso o usuário não fa\c{c}a a inser\c{c}ão destes valores. Inteiros aleatórios são gerados para simular a probabilidade P de existir ilhas ou não, seu tamanho será aleatoriamente gerado dentro dos limites da terceira por\c{c}ão da grade. Com intuito de evitar o encontro de uma das margens com uma dada ilha é considerado que \textbf{a distância mínima entre os dois deve ser de 2 pontos}
Aplica-se o método \textit{velocidadeDaAguaPrimeiraLinha} a cada ponto de água que não seja vizinho a uma margem. Esta velocidade é a soma de um número aleatório no invervalo $[-0.5,0.5]$ somado a velocidade do ponto anterior apenas se o ponto anterior não for margem. A fun\c{c}ão também lida com casos especiais, como valores aleatórios negativos e pontos anteriores a vizinhos de margens. Matemáticamente a opera\c{c}ão relizada é a seguinte:

\begin{equation}
v_i = v_{i-1} + \lambda \textbf{\, , } \lambda \in [-0.5,0.5]
\end{equation}

Ao fim de todos estes passos, de gera\c{c}ão à inser\c{c}ão, é necessário ajustar o fluxo da água na linha. Primeiro suave-se a velocidade dos pontos	 para que a diferen\c{c}a entre eles não seja tão discrepante, isto é feito em \textit{suavizaVelocidade}, e então a velocidade dos pontos é normalizada em \textit{normaliza} com o uso da formula de interpola\c{c}ão: 

\begin{equation}
\phi = \sum_{i=0}^{N} v_i
\end{equation}
\begin{equation}
v_i \leftarrow v_i\cdot\dfrac{\Phi}{\phi}
\end{equation}
\\
onde $\Phi$ é o fluxo inserido pelo usuário.

\subsection{proximaLinha}

A gera\c{c}ão das linhas subsequentes funciona de modo semelhante a fun\c{c}ão primeiraLinha. Cada nova linha utiliza a anterior como base para sua cria\c{c}ão. As diferen\c{c}as residem nos métodos que lidam com as margens e com a velocidade das linhas. O método \textit{aleatorizaMargem} analisa as margens anteriores e sorteia um valor inteiro no intervalo $[-1,1]$. Este valor é somado as margens da linha anterior e atribuindo-se à nova, respeitando sempre o limite das margens, isso garante que a mudan\c{c}a entre a margem anterior e atual seja suave. Para gera\c{c}ão de velocidades na nova linha o método \textit{velocidadeProximaLinha} é empregado. Com intuito de preservar a suavidade da velocidade entre pontos é tomado como base a velocidade do ponto anterior e realizado o produto desta com um valor aleatório real no interválo $[0.9,1.1]$, e no caso do ponto anterior ter velocidade zero, realiza-se um outro cálculo.

\section{Grade}
Esta interface concentra o método de cria\c{c}ão da matriz e dos frames utilizados durante o programa. O método \textit{initGrade} aloca dinamicamente uma matriz de tamanho $altura \times largura$, as quais tem valor default de 30 por 100 respetivamente. Em \textit{criaPrimeiroFrame} cria-se a primeira imagem que será impressa, a constru\c{c}ão das linhas acontece de baixo para cima, indo da ultima linha da matriz até a primeira. A mesma ideia é utilizada em \textit{criaProximoFrame}, entretanto nesta é criada apenas uma nova linha e inserida na posi\c{c}ão $(indice + alturaDaGrade -1)\text{ \, } mod \text{ \, } alturaDaGrade$ . Um método de desaloca\c{c}ão de memória, \textit{freeGrade}, foi implementado para situa\c{c}ões futuras nas quais a execu\c{c}ão do programa não seja interrompida pelo comando CTRL+C.

\section{PrintASCII}
Esta interface lida com a impressão dos frames na tela para o usuário. No método \textit{outputArray} seleciona-se a linha que deve ser impressa e chama-se o método \textit{outputLine} que cuida da impressão dos caracteres. Uma particularidade ocorre no método \textit{outputArray}, utiliza-se rela\c{c}ões de congruência (mod) para implementar o buffer circular de impressão, a primeira linha a ser impressa será a ultima que foi criada na interface \textit{Grade}.

\section{Util}
Implementado nesta interface está a fun\c{c}ão \textit{getArgs} que faz a leitura dos parâmetros passado pelo usuário e atribuido as suas respectivas variáveis. Caso seja passado algum parâmetro desconhecido é mostrado ao usuário uma tabela com as entradas possíveis, é perguntado se este quer realizar um novo input ou se deseja abortar o programa. Após a chamada de \textit{getArgs} um outro método, 
\textit{corrigeArgs}, é chamado para analisar as entradas fornecidas pelo usuário com intuito que corrigir possíveis parâmetros que comprometam a correta execu\c{c}ão do programa.

\section{main}
Esta é a interface client do programa, nela são chamadas todas outras fun\c{c}ões que realizam o trabalho bruto de cria\c{c}ão, gerenciamento e impressão da simula\c{c}ão. No anexo 3 está descrito o fluxo de informa\c{c}ões trocadas entre o cliente e suas diversas interfaces. Para encerar a execu\c{c}ão do programa o usuário deve pressionar o comando CTRL+D, abortando assim o programa. Pelo fato do programa ser abortado sua execu\c{c}ão nunca chegará ao final e com isso a fu\c{c}ão de desaloca\c{c}ão de memória \textit{freeGrade} não é chamada, decidiu-se declarar e implementar esta fun\c{c}ão como uma política de boas práticas de programa\c{c}ão, já que o cliente solicita a interface uma aloca\c{c}ão de memória é obriga\c{c}ão daquele a sua desaloca\c{c}ão, mesmo que não seja chamada a fun\c{c}ão já existe para futuros estágios em que será solicitada.

\section{teste}
Esta interface client é semelhante a main, diferindo-se na sua utilidade. Este client faz a execu\c{c}ão dos métodos do programa com intuito de testar se as saídas do programa são concretas e confiáveis. A principal fun\c{c}ão de teste é \textit{calculaVariacoes} que desempenha diversos cálculos. São gerados 50 frames e aplicado o método \textit{calculaVariacoes} procurando os seguintes valores:
\begin{enumerate}
\item[•]{Número de linhas com fluxo correto}
\item[•]{Velocidade máxima que um ponto de água adquiriu}
\item[•]{Velocidade mínima que um ponto de água adquiriu}
\item[•]{Velocidade média entre todos os pontos criados}
\item[•]{Comprimento máximo da margem esquerda}
\item[•]{Comprimento mínimo da margem esquerda}
\item[•]{Comprimento médio da margem esquerda}
\item[•]{Comprimento máximo da margem direita}
\item[•]{Comprimento mínimo da margem direita}
\item[•]{Comprimento médio da margem direita}
\end{enumerate}
Por fim esses valores são impressos na tela de forma diagramada.

\section{rotinaTeste}
Esta interface concentra os dois métodos de teste, \textit{testaCorrecao} e \textit{calculaVariacoes}. Na primeira testa-se se o fluxo de uma linha é igual o fluxo desejado pelo usuário e na segunda realiza-se os outros cálculos e compara\c{c}ões, como o comprimento maior, menor e médio das margens, velocidades máxima, mínima e média. A fun\c{c}ão também retorna quantas linhas do frame análisado passaram no teste do fluxo.

\section{Implementa\c{c}ão grafica}
Na parte B do exercício programa 2 a nova tarefa a ser implementada é a interface gráfica do programa utilizando a biblioteca gráfica Allegro. Tomando mão da modularização dos métodos foi possível manter a mesma implementação da etapa anterior fazendo com que a unica alteração substancial dentro do ep seja na dinâmica de impressão que agora é feita de forma gráfica e não mais textual. Descreveremos completamente as mudanças e melhorias realizadas para que o programa ganhasse o aspecto desejado.

\subsection{main}
Estruturalmente main continua como na fase anterior, como seu fluxo de funcionamento. Existe agora uma nova funçcão /testeit{STinitAllegro} que inicializa as funcionalidades utilizadas pelo Allegro para simulação gráfica. No looping principal é executado os métodos de criação da grade, verificação de eventos vindos do teclado e impressão gráfica da grade. Decidimos por implementar uma discreta navegação do barco ao longo do rio para nos nortear quanto o desempenho da velocidade do barco em relação a animação do fundo. Para que a execução seja encerrada implemetamos a tecla \textbf{ESC} que aborta o programa, em seguinda desaloca-se a memória utilizada e encerra-se o programa.
\subsection{PrintAllegro}
Em PrintAllegro temos a \textit{OutputArray} que cuida de toda lógica e impressão da parte gráfica. Caso a linha anterior seja menor que a atual um triangulo é impresso para suavizar as diferenças das margens, caso a anterior seja maior e a atual menor o mesmo também é feito, só o triângulo que muda de orientação, no caso de margens com mesmo comprimento uma linha reta é desenhada. O complemento do triângulo, que seria água, é desenhado logo em seguida, fazendo que não haja pontos sem pixels. Essa metodologia é empregado para as duas margens, no caso de ilhas e água uma simples impressão é realizada.
Uma elipse foi escolhida para representar a posição do barco, sua posição pode ser manipulada com as setas do teclado simulando quase que totalmente o funcionamento do jogo. Cada caracter textual tem como valor default 5x5 pixels, mas este valor pode ser alterado pelo usuário, entretando uma alta densidade de pixels torna a execução muito mais lenta.

\section{Execu\c{c}ão do programa}
\subsection{Makefile}
No makefile do programa existem três receitas de compila\c{c}ão que geram três alvos diferentes:
\begin{enumerate}
\item[-]{make - Gera o executável ep2 que é o programa de simula\c{c}ão do rio propriamente dito}
\item[-]{make teste - Gera o executável teste que realiza as simula\c{c}ões de teste do rio e a gera\c{c}ão do relatório}
\item[-]{make clean - Limpa os arquivos objeto e executáveis gerados pela execu\c{c}ão das outras receitas}
\end{enumerate}
\subsection{Parâmetros de execu\c{c}ão}
A simula\c{c}ão do rio pode ser feita sem que o usuário entre com qualquer parâmetro de defini\c{c}ão, gra\c{c}as a valores de default implementados. Entretando, caso seja do desejo do usuário pode-se entrar com diversos parâme-tros de defini\c{c}ão para execu\c{c}ão da simula\c{c}ão ou do teste, os quais são os seguintes:
\begin{enumerate}
\item[]{-b  $\rightarrow$ Velocidade do barco}
\item[]{-l  $\rightarrow$ Largura do Rio}
\item[]{-s  $\rightarrow$ Semente para o gerador aleatorio}
\item[]{-f  $\rightarrow$ Fluxo da agua}
\item[]{-pI $\rightarrow$ Probabilidade de haver obstaculos}
\item[]{-dI $\rightarrow$ Distancia minima entre obstaculos}
\item[]{-lM $\rightarrow$ Limite das margens}
\item[]{-v  $\rightarrow$ Verbose}
\item[]{-D  $\rightarrow$ Tamanho de cada pixel}
\end{enumerate} 


Foto da execu\c{c}ão do programa


\newpage
\begin{center}
\textbf{\text{\Large  Anexo 1 - Fluxo de funcionamento primeiraLinha}}
\begin{tikzpicture}[>=stealth',shorten >=1pt,node distance=3cm,on grid,initial/.style    ={}]
  \node[state]          (P)                        {$pL$};
  \node[state]          (B) [ above =of P]    {$alePM$};
  \node[state]          (E) [above right =of P]    {$marE$};
  \node[state]          (D) [right =of P]    {$marD	$};
  \node[state]          (M) [below right =of P]    {$iIlha	$};
  \node[state]          (K) [below =of P]    {$velAPL$};
  \node[state]          (F) [below left =of P]    {$sVel$};
  \node[state]          (C) [left =of P]    {$nLz$};
\tikzset{mystyle/.style={<->,double=orange}} 
\tikzset{every node/.style={fill=white}} 
\path (P)     edge [mystyle]    node   {$1$} (B)
      (P)     edge [mystyle]    node   {$2$} (E) 
      (P)     edge [mystyle]    node   {$3$} (D) 
      (P)     edge [mystyle]    node   {$4$} (M) 
      (P)     edge [mystyle]    node   {$5$} (K)
      (P)     edge [mystyle]    node   {$6$} (F)
      (P)     edge [mystyle]    node   {$7$} (C);

\textbf{\text{ primeiraLinha - Legenda }}
\end{tikzpicture}
\end{center}
\textbf{\text{  primeimaLinha - Legenda }}
\begin{enumerate}
\item[-]{pL - primeiraLinha}
\item[-]{alePM - aleatorizaPrimeiraMargem}
\item[-]{marE - margemEsquerda}
\item[-]{marD - margemDireita}
\item[-]{iIlha - InsereIlha}
\item[-]{velAPL - velocidadeDaAguaPrimeiraLinha}
\item[-]{sVel - suavizaVelocidades}
\item[-]{nLz - normaliza}
\end{enumerate}


\newpage
\begin{center}
\textbf{\text{\Large  Anexo 2 - Fluxo de funcionamento proximaLinha}}
\begin{tikzpicture}[>=stealth',shorten >=1pt,node distance=3cm,on grid,initial/.style    ={}]
  \node[state]          (P)                        {$pxL$};
  \node[state]          (B) [ above =of P]    {$aleM$};
  \node[state]          (C) [right =of B]    {$marD$};
    \node[state]          (M) [above right =of C]    {$marE$};
  \node[state]          (F) [above left   =of P]    {$iIlha$};
  \node[state]          (G) [right =of P]    {$velpxL$};
  \node[state]          (R) [below =of G]    {$rRand$};
  \node[state]          (S) [below =of P]    {$sVel$};
  \node[state]          (H) [below left =of P]    {$nLz$};
\tikzset{mystyle/.style={<->,double=orange}} 
\tikzset{every node/.style={fill=white}} 
\path (P)     edge [mystyle]    node   {$1$} (B)
      (B)     edge [mystyle]    node   {$2$} (M)
      (B)     edge [mystyle]    node   {$3$} (C)  
      (P)     edge [mystyle]    node   {$4$} (F)
      (P)     edge [mystyle]    node   {$5$} (G)
      (G)     edge [mystyle]    node   {$6$} (M)
      (G)     edge [mystyle]    node   {$7$} (C)
      (G)     edge [mystyle]    node   {$8$} (R)
      (P)     edge [mystyle]    node   {$9$} (S)
      (P)     edge [mystyle]    node   {$10$} (H);
\end{tikzpicture}
\end{center}
\textbf{\text{  proximaLinha - Legenda }}
\begin{enumerate}
\item[-]{pxL - proximaLinha}
\item[-]{aleM - aleatorizaMargem}
\item[-]{marE - margemEsquerda}
\item[-]{marD - margemDireita}
\item[-]{iIlha - InsereIlha}
\item[-]{velpxL - velocidadeProximaLinha}
\item[-]{rRand - realRandomico}
\item[-]{sVel - suavizaVelocidades}
\item[-]{nLz - normaliza}
\end{enumerate}



\newpage
\begin{center}
\textbf{\text{\Large  Anexo 3 - Fluxo de funcionamento da interface gráfica}}
\begin{tikzpicture}[>=stealth',shorten >=1pt,node distance=3cm,on grid,initial/.style    ={}]
  \node[state]          (P)                         {$PrtF$};
  \node[state]          (B) [ left =of P]           {$FrtF$};
  \node[state]          (C) [left =of B]            {$STall$};
  \node[state]          (F) [below right   =of P]    {$inK$};
  \node[state]          (M) [below left =of F]     {$NxtF$};
  \node[state]          (G) [above left =of M]           {$outK$};
  \node[state]          (R) [below left =of G]           {$endE$};
\tikzset{mystyle/.style={->,double=orange}} 
\tikzset{every node/.style={fill=white}} 
\path (C)     edge [mystyle]    node   {$1$} (B)
      (B)     edge [mystyle]    node   {$2$} (P)
      (P)     edge [mystyle]    node   {$3$} (F)  
      (F)     edge [mystyle]    node   {$4$} (M)
      (M)     edge [mystyle]    node   {$5$} (G)
      (G)     edge [mystyle]    node   {$6$} (P)
      (G)     edge [mystyle]    node   {$7$} (R);
\end{tikzpicture}
\end{center}
\textbf{\text{  Fluxo de funcionamento da interface gráfica - Legenda }}
\begin{enumerate}
\item[-]{STall - Inicialização do allegro}
\item[-]{FrtF - Criação do primero frame}
\item[-]{PrtF - Impressão do frame}
\item[-]{inK - Leitura de um evento do teclado}
\item[-]{NxtF - Criação dos novos frames}
\item[-]{outK - Execução de um evento  do teclado}
\item[-]{endE - Fim da execução}
\end{enumerate}




\newpage
\begin{center}
\textbf{\text{\Large  Anexo 4 - Fluxo de funcionamento main}}
\begin{tikzpicture}[>=stealth',shorten >=1pt,node distance=3cm,on grid,initial/.style    ={}]
  \node[state]          (P)                        {$main$};
  \node[state]          (B) [ above left =of P]    {$gArgs$};
  \node[state]          (Q) [ left =of P]    {$cArgs$};
  \node[state]          (M) [above =of P]    {$initG$};
  \node[state]          (D) [above right =of P]    {$cPrFrame$};
  \node[state]          (F) [below left =of P]    {$nanoS$};
  \node[state]          (N) [below right =of F]    {$clrScr$};
  \node[state]          (C) [right =of N]    {$OutA$};
  \node[state]          (E) [below right =of D]    {$cPxFrame$};

\tikzset{mystyle/.style={<->,double=orange}} 
\tikzset{every node/.style={fill=white}} 
\path (P)     edge [mystyle]    node   {$1$} (B)
      (P)     edge [mystyle]    node   {$2$} (Q) 
      (P)     edge [mystyle]    node   {$3$} (M) 
      (P)     edge [mystyle]    node   {$4$} (D)
      (P)     edge [mystyle]    node   {$5$} (C)
      (P)     edge [mystyle]    node   {$7$} (E) 
      (P)     edge [mystyle]    node   {$8$} (F)
      (P)     edge [mystyle]    node   {$6$} (N);
      
      \tikzset{mystyle/.style={->,double=orange}} 
\tikzset{every node/.style={fill=white}} 
\path
      (E)     edge [mystyle]    node   {$loop$} (C)
      (C)     edge [mystyle]    node   {$loop$} (N) 
      (N)     edge [mystyle]    node   {$loop$} (F)
      (F)     edge [mystyle]    node   {$loop$} (E);

\textbf{\text{ primeiraLinha - Legenda }}
\end{tikzpicture}
\end{center}
\textbf{\text{  main - Legenda }}
\begin{enumerate}
\item[-]{gArgs - getArgs}
\item[-]{cArgs - corrigeArgs}
\item[-]{initG - initGrade}
\item[-]{cPrFrame - criaPrimeiroFrame}
\item[-]{nanoS - nanoSleep}
\item[-]{clsScr - clearScreen}
\item[-]{OutA - OutputArray}
\item[-]{cPxFrame - criaProximoFrame}
\end{enumerate}

\end{document}